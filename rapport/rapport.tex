\documentclass[11pt,a4paper]{article}

\usepackage[utf8]{inputenc}
\usepackage[francais]{babel}
\usepackage[T1]{fontenc}

\usepackage{graphicx}
\usepackage[margin=3cm]{geometry}
\usepackage{hyperref} %Pour les liens dans le sommaire
\usepackage{lscape} %Pour le format paysage
\usepackage{fancyhdr} %Pour le format du rapport
\usepackage{xcolor} %Pour l'utilisation de couleur dans le texte

\usepackage{amsmath}
\usepackage{amsfonts}
\usepackage{amssymb}


\definecolor{myGreen}{rgb}{0,0.46,0.38} %#007561
\definecolor{myKaki}{rgb}{0.60,0.45,0} %#997200
\definecolor{myBlue}{rgb}{0,0.45,0.68} %#0073ad
\definecolor{myPurple}{rgb}{0.51,0.25,0.76} %#8241c2
\definecolor{myOrange}{rgb}{0.95,0.50,0.05} %#f17f0d
\definecolor{myDarkBlue}{rgb}{0,0,0.63} %#0000a0
\definecolor{myDarkRed}{rgb}{0.44,0,0} %#700000
\definecolor{myRed}{rgb}{0.73,0.04,0.11} %e00909
\definecolor{myGrey}{rgb}{0.439,0.439,0.439} %707070


\author{Vincent BEUGNET - Adlane LADJAL}
\title{Devoir de GLOA \\ Application de commande de satellite}


\begin{document}

\makeatletter
\begin{titlepage}
	\centering
	\includegraphics[width=0.25\textwidth]{logo_supgalilee.jpg}
	\hfill
	\includegraphics[width=0.25\textwidth]{logo-paris13_bis.png} \\
    \vspace{5cm}
       {\LARGE \textbf{\@title}} \\
    \vspace{2em}
        {\large \@author }\\
    \vspace{1em}
        {\textit{\@date}} \\
    \vspace{2em}
    		\includegraphics[width=0.35\textwidth]{logo-esa.jpg}\\
    	\vspace{2em}
    		{Enseignants : Christine CHOPPY -- John CHAUSSARD} \\
    \vfill
\end{titlepage}


%insertion de la page blanche
\newpage
~
\newpage

%Table des matieres
\renewcommand{\contentsname}{Sommaire}
\tableofcontents

\newpage

\pagestyle{empty}

\section{Introduction}
Le présent devoir a été réalisé par Vincent Beugnet 
et Adlane LADJAL, au cours de notre deuxième année 
de la  formation d'Ingénieur Informatique à 
Sup Galilée. Il a été réalisé dans le cadre du cours 
Génie Logiciel Avancée (GLOA). Ce cours est la continuité
du cours de Modélisation des Systèmes Informatiques
que nous avons eu en première année des études d'ingénieur.
Le but de ce cours est de s'intéresser aux méthodes pour
décrire le fonctionnement d'un logiciel afin qu'ils répondent
au mieux aux besoins utilisateurs.

Ce rapport de projet décrit la réflexion que nous avons pu 
avoir pour réaliser le système informatique suivant.

Nous devions concevoir le système informatique permettant
de gérer une application pour l’\textit{European Space Agency}
(ESA). L’ESA veut pouvoir envoyer un satellite vers la ceinture
d'astéro\"ides pour effectuer toutes sortes de mesures.
Pour se faire elle a besoin de sponsors afin de financer une
partie du projet. Mais elle veut aussi sensibiliser : elle veut
pouvoir alors offrir la possibilité à quiconque d'acheter un
surnom pour un objet céleste, en plus de son nom scientifique.

Ce fut un long travail, où des choix ont dû être fait, 
comme par exemple la représentation ou non de certaines 
choses demandées dans le cahier des charges. Il a fallu 
le lire et le relire afin de rester le 
plus cohérent possible.

\newpage


\section{Définition de la borne du système}


\newpage

%%%%%%%%%%%%%%%%%%%%%%%%%%%%%%%%%%%%%%%%%%%%%%%%%%%%%%%%%

\section{Les acteurs}

\subsection{Tableau acteur -- r\^ole -- description}



\subsection{Tableau acteur -- objectifs}

\newpage

%%%%%%%%%%%%%%%%%%%%%%%%%%%%%%%%%%%%%%%%%%%%%%%%%%%%%%%%%

\section{Les cas d’utilisations}

\subsection{Le diagramme de cas d’utilisations}

\vspace{4cm}

\includegraphics[scale=0.25]{det.png}

\newpage

\subsection{Développement des cas d’utilisations}


\newpage

%%%%%%%%%%%%%%%%%%%%%%%%%%%%%%%%%%%%%%%%%%%%%%%%%%%%%%%%%

\section{Schémas de collaboration}

\subsection{La demande de positionnement}


\newpage

\subsection{L'achat d'un surnom}

\newpage

%%%%%%%%%%%%%%%%%%%%%%%%%%%%%%%%%%%%%%%%%%%%%%%%%%%%%%%%%

\section{L'int\'eraction \'el\'ementaire de la demande de positionnement}


\newpage



%%%%%%%%%%%%%%%%%%%%%%%%%%%%%%%%%%%%%%%%%%%%%%%%%%%%%%%%%

\section{L'observateur sur le rang d'un sponsor}

\newpage

%%%%%%%%%%%%%%%%%%%%%%%%%%%%%%%%%%%%%%%%%%%%%%%%%%%%%%%%%

\section{Conclusion}
Ce projet nous a permis d'apprendre à créer des diagrammes de manière informatique. Pour réaliser l'ensemble des diagrammes nous avons utilisé le logiciel StarUML. Celui-ci permet de créer un large éventail de diagramme incluant ceux vu en cours. Nous avons également appris à utiliser le langage LaTeX en réalisant ce rapport sur le site sharelatex.com, qui permet de travailler à deux simultanément sur un projet LaTeX.

StarUML est un très bon logiciel néanmoins nous avons eu du mal à réaliser le diagramme de communication dessus car nous n'avons pas réussi à obtenir la synthaxe \textit{1.2:} . Pour résoudre ce problème nous avons supprimé le nom par defaut, et nous avons rajouté une zone de texte libre à la place.

Grâce à ce projet nous savons réaliser de nombreux diagrammes correspondant à un cas donné. Nous savons quels sont les apports de chaque diagramme selon ce que nous avons envie d'exposer. Nous avons également appris à utiliser le bon diagramme et les bonnes informations pour construire un autre type de diagramme.

Lors de la réalisation du projet, ce qui nous a posé le plus de difficulté, c'est le niveau de détail à appliquer à chaque diagramme pour que celui-ci retranscrive le plus d'informations tout en restant lisible.
\end{document}
